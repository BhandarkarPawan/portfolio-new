%-------------------------
% Resume in Latex
% Author: Panambur Pawan Bhandarkar 
% License: MIT
%------------------------

\documentclass[letterpaper]{article}

\usepackage{latexsym}
\usepackage[empty]{fullpage}
\usepackage{titlesec}
\usepackage{marvosym}
\usepackage[usenames,dvipsnames]{color}
\usepackage{verbatim}
\usepackage{enumitem}
\usepackage{fancyhdr}
\usepackage[hidelinks]{hyperref}


\pagestyle{fancy}
\fancyhf{}
\fancyfoot{}
\setlength{\footskip}{5pt}
\renewcommand{\headrulewidth}{0pt}
\renewcommand{\footrulewidth}{0pt}

\addtolength{\oddsidemargin}{-0.5in}
\addtolength{\evensidemargin}{-0.5in}
\addtolength{\textwidth}{1in}
\addtolength{\topmargin}{-.5in}
\addtolength{\textheight}{1.0in}

\urlstyle{rm}

\raggedbottom
\raggedright
\setlength{\tabcolsep}{0in}

\titleformat{\section}{
  \vspace{-3pt}\scshape\raggedright\large\bfseries{}
}{}{0em}{}[\color{black}\titlerule \vspace{-7pt}]

%-------------------------
\newcommand{\resumeItem}[2]{
  \item\small{
    \textbf{#1}{: #2 \vspace{-2pt}}
  }
}

\newcommand{\resumeItemWithoutTitle}[1]{
  \item\small{
    {#1 \vspace{-2pt}}
  }
}

\newcommand{\resumeSubheadingWithoutTitle}[2]{
  \begin{tabular*}{\textwidth}{l@{\extracolsep{\fill}}r}
        \textbf{\textit{#1}} & \textit{\small #2} \\
    \end{tabular*}\vspace{-14pt}
}
\newcommand{\resumeSubheading}[4]{
    \begin{tabular*}{\textwidth}{l@{\extracolsep{\fill}}r}
        \textbf{#1} & #2 \\
        \textbf{\textit{#3}} & \textit{\small #4} \\
    \end{tabular*}\vspace{-10pt}
}



\newcommand{\shortSection}[1]{
    \vspace{-6pt}
    \section{#1}
}

\newcommand{\educationHeading}[5]{
    \begin{tabular*}{\textwidth}{l@{\extracolsep{\fill}}r}
        \textbf{#1} & {#2} \\
        \textit{\small #3} & \textit{\small #4} \\
    \end{tabular*}
    \small{\textbf{Courses :}{#5}}
}

\newcommand{\projectHeading}[3]{
  \begin{tabular*}{\textwidth}{l@{\extracolsep{\fill}}r}
        \textbf{#1} 
        \hspace{-2pt} $\vert$ \hspace{-2pt} \small{\textit{#2}} 
        & 
        \textit{#3} \\
    \end{tabular*}\vspace{-2pt}
}


\newcommand{\projectHeadingWithLink}[4]{
  \begin{tabular*}{\textwidth}{l@{\extracolsep{\fill}}r}
        \href{#4}{\color{blue}{#1}}
        \hspace{-2pt} $\vert$ \hspace{-2pt} \small{\textit{#2}} 
        & 
        \textit{#3} \\
    \end{tabular*}\vspace{-2pt}
}

\newcommand*{\skill}[2]{
  \textbf{#1 : }#2 \\
  \vspace{1pt}
}


\newcommand{\resumeSubItem}[2]{\resumeItem{#1}{#2}\vspace{-4pt}}
\renewcommand{\labelitemii}{$\circ$}

\newcommand{\resumeSubHeadingListStart}{\begin{itemize}[leftmargin=*]}
\newcommand{\resumeSubHeadingListEnd}{\end{itemize}}
\newcommand{\resumeItemListStart}{\begin{itemize}}
\newcommand{\resumeItemListEnd}{\end{itemize}}

%-------------------------------------------

\begin{document}

%----------HEADING-----------------
\begin{tabular*}{\textwidth}{l@{\extracolsep{\fill}}r}

    \textbf{{\LARGE Pawan Bhandarkar}}\\
    Mountain View, CA & Portfolio:\href{https://www.bhandarkar.me/}{ \color{blue}{https://bhandarkar.me}} \\
    Email : \href{mailto:pawan@bhandarkar.me}{pawan@bhandarkar.me}   &   Github:\href{https://github.com/BhandarkarPawan}{ \color{blue}{https://github.com/BhandarkarPawan}}\\
    Mobile : +1(650)537-7341 &  Linkedin:\href{https://www.linkedin.com/in/bhandarkar/}{ \color{blue}{https://www.linkedin.com/in/bhandarkar}} \\

\end{tabular*}


%-----------SUMMARY-----------------
% \shortSection {Summary}
% \small Full-stack engineer experienced (2+ years) in building \textbf{responsive}, \textbf{accessible} web applications. I write \textbf{clean, modular}, \textbf{well-documented} and testable code that is easy for teams to \textbf{understand, extend}, and \textbf{integrate}. 



%-----------EDUCATION-----------------
\shortSection{Education}
\educationHeading
{Carnegie Mellon University}{Mountain View, CA}
{Master of Science in Software Engineering; \textbf{GPA: 4.0/4.0} }{Dec 2023}{
    Cloud Computing,
    Foundations of Software Engineering,
    Software Requirements \& Interaction Design,
    % Software Verification \& Testing,
    Software Testing
    % Data Science
}

\vspace{5pt}

\educationHeading
{NMAM Institute of Technology }{Nitte, India}
{Bachelor of Computer Science;  \textbf{GPA: 9.8/10} (top 1\%, Class of 2020) }{Apr 2020}{
    Data Structures \& Algorithms,
    Object Oriented Modelling \& Design,
    Operating Systems,
    RDBMS
}


%-------- SKILLS------------
\shortSection{Skills}
\skill {Languages \hspace{4pt}}{JavaScript, TypeScript, Python, Java, C++, C\#, Go, GraphQL}
\skill {Tools\hspace{32pt}}{AWS, ActiveMQ, Azure, Docker, Kafka, Redis, Terraform, Kubernetes, Linux, gRPC, New Relic}
\skill {Frameworks}{React, Flask, PostgreSQL, Flutter, Redux, NodeJS, Express, MongoDB}
\skill {Soft Skills\hspace{10pt}}{Communication (verbal and written), Leadership, Collaboration, Mentorship}



%-----------EXPERIENCE-----------------
\shortSection{Experience}
\resumeSubheading
{TuneIn Radio}{San Francisco, USA}
{Software Engineer Intern, Platform}{Jun 2023 - Aug 2023}
\vspace{2pt}
\resumeItemListStart

% \resumeItemWithoutTitle{Pioneered subscription-based APIs via WebSockets through Test-Driven Development on the GraphQL API gateway backed by microservices using gRPC communication in Go}
% \resumeItemWithoutTitle{Thrived in a fast-paced environment, pioneering subscription-based APIs via WebSockets through Test-Driven Development on the GraphQL API gateway backed by microservices using gRPC communication in Go}
\resumeItemWithoutTitle{Spearheaded the development of subscription-based APIs using WebSockets, enhancing real-time data delivery for up to 7.5M monthly users. Utilized Test-Driven Development in Go to ensure a robust GraphQL API gateway}

% \resumeItemWithoutTitle{Built data communication pipelines using Apache ActiveMQ and Redis Streams (AWS ElastiCache)}
\resumeItemWithoutTitle{Engineered high-throughput data communication pipelines using Apache ActiveMQ and Redis Streams, processing over 2 million messages per day, with latencies under 30 ms, on AWS ElastiCache}

% \resumeItemWithoutTitle{Wrote comprehensive test suites and load-tested the microservice to ensure production-readiness}
\resumeItemWithoutTitle{Achieved 95\% code coverage with test suites and load-tested for 100,000 concurrent requests to ensure production readiness}

% \resumeItemWithoutTitle{Deployed microservices with Infrastructure as Code using Terraform and Kubernetes and monitored them using New Relic}
\resumeItemWithoutTitle{Automated microservice deployment using Terraform and Kubernetes. Maintained 99.9\% uptime with New Relic monitoring}

\resumeItemWithoutTitle{Expanded the C\# (.NET Core) monolith codebase with support for live streaming metadata for over 120,000 radio stations}
\resumeItemWithoutTitle{Collaborated with Product and Data Engineering teams to plan and prioritize related Jira tickets }
\resumeItemListEnd

\vspace{5pt}
\resumeSubheading
{Team AIBOD Inc}{Fukuoka City, Japan}
{Software Engineer 2}{Nov 2021 - Apr 2022}
\vspace{2pt}
\resumeItemListStart

% \resumeItemWithoutTitle{Developed a component library with React v16, CSS, and TypeScript for an unmanned store}
\resumeItemWithoutTitle{Created a React component library, reducing UI variants by 50\% to enhance consistency in the POS for an unmanned store}

% \resumeItemWithoutTitle{Led the development of a payment microservice with Python, Flask, and PostgreSQL with HTTPS/SSL for enhanced security}
\resumeItemWithoutTitle{Led the development of a payment microservice, handling 200 transactions per hour, with gRPC, Flask and PostgreSQL}

% \resumeItemWithoutTitle{Engineered a full-stack information management system using Python, SQL, GraphQL, and React v16}
\resumeItemWithoutTitle{Engineered an information management system for serving up to 150,000 users daily,  with React, Python, SQL and GraphQL}

\resumeItemWithoutTitle{Deployed applications using Docker and AWS (ECR + ECS) and configured CloudWatch for monitoring}
\resumeItemListEnd

\vspace{2pt}
\resumeSubheadingWithoutTitle
{Software Engineer 1}{Feb 2020 - Oct 2021}
\vspace{0pt}
\resumeItemListStart
\resumeItemWithoutTitle{Slashed total API calls by 80\% by pioneering the use of Redux, contributing to a 20\% improvement in app load times}
\resumeItemWithoutTitle{Integrated 30+ RESTful APIs leveraging Apiary and Postman, resulting in enhanced app functionality}
\resumeItemWithoutTitle{Set up a Jenkins CI/CD pipeline that automated 100+ weekly builds, while utilizing AWS services like ECR, ECS, S3, and AWS Fargate for seamless DevOps}
\resumeItemWithoutTitle{Developed a Unet-based AI image annotation tool, elevating data pre-processing rates from 20 to 200+ images/hour}
\resumeItemListEnd

% \vspace{2pt}
% \resumeSubheadingWithoutTitle
% {AI Engineer Intern}{Feb 2020 - Jun 2020}
% \vspace{0pt}
% \resumeItemListStart
% \resumeItemWithoutTitle{Applied \textbf{Feature Engineering} to improve the accuracy of a \textbf{KNN-supervised classifier} by 7\%, resulting in more effective classification of products in an unmanned store}
% \resumeItemWithoutTitle{Experimented with \textbf{BERT} using \textbf{PyTorch} for \textbf{Named Entity Recognition} to automate support request routing in an apartment intercom system and integrated it with the \textbf{Express API Server} to provide AI as a Service}
% \resumeItemWithoutTitle{Expertly resolved more than 20 bugs in the API server using \textbf{NodeJS and Typescript}, resulting in improved stability and performance of the server, also contributed to ongoing full-stack applications}
% \resumeItemListEnd

%-----------PROJECTS-----------------



\shortSection{Projects}
\vspace{3pt}
\projectHeadingWithLink {Kanban Task Manager}{React, EC2, ECR, Docker, NodeJS, MongoDB}{Jan 2023 - Present}{https://kanbhan.com}
\resumeItemListStart

% \resumeItemWithoutTitle{Developed a highly accessible Kanban Task Manager using Client-Server architecture with React Hooks and Context API focussing on A11y, Semantic HTML, and ARIA labels, which ensured usability with screen readers such as Apple Voice-Over. Generated clear API documentation with API Blueprint and Postman}
\resumeItemWithoutTitle{Developed a responsive web app with pixel perfect user interfaces based on Figma designs. Used Client-Server architecture with dynamic theming and drag-and-drop UI, focusing on A11y, ARIA and Semantic HTML. Deployed to AWS EC2 using Docker and ECR with a CI/CD pipeline using Github Actions. Handled security with HTTPS/SSL on the AWS Load Balancer}
\resumeItemListEnd


\vspace{3pt}
\projectHeading {Incident Response}{React, CloudFront, Docker, Socket.IO, MongoDB, Jenkins}{Feb 2023 - May 2023}
\resumeItemListStart

% Backend
% \resumeItemWithoutTitle{Developed a communication platform for citizens and first responders during emergencies using Express.js, Socket.IO, and Mongoose. Refactored codebase and managed development in a scrum-of-scrums setting. Deployed the app using Docker, Jenkins, and AWS}
% Frontend
\resumeItemWithoutTitle{Led the frontend development of a mobile-first communication platform for citizens and first responders during emergencies. Used Ant Design, Figma and Storybook for the UI, refactored the codebase and advised the team on best practices in UI development and accessibility. Deployed the app to S3 using Jenkins and CloudFront}
\resumeItemListEnd


\vspace{3pt}
\projectHeading {Emergency Social Network}{HTML, CSS, TypeScript, NodeJS, MongoDB}{Aug 2022 - Dec 2022}
\resumeItemListStart
\resumeItemWithoutTitle{Built an emergency communication system for real-time SOS messaging during earthquakes with an MVC architecture. Utilized SCRUM and Kanban agile practices, object-oriented analysis (OOA) and test-driven development (TDD) in a fast-paced environment to create a user-friendly system with JSON-based RESTful APIs}
\resumeItemListEnd


% \vspace{3pt}
% \projectHeading {Re:use}{React, Typescript, NodeJS, MongoDB, Figma}{Aug 2022 - Dec 2022}
% \resumeItemListStart
% \resumeItemWithoutTitle{Led the development of the MVP for a Resource Management application in the donation industry by generating style guides in Figma, and converting user requirements into working software using the MERN Stack}
% \resumeItemWithoutTitle{\textbf{Ideated} a solution for problems faced by the donation industry through \textbf{usability studies} and \textbf{contextual inquiries} with Stakeholders. Generated \textbf{affinity diagrams}, \textbf{personas}, and \textbf{story boards} to envision a solution using the Dual-Track Agile process. Worked as an interaction designer to drive user research/interviews, create a core product vision and build a MVP}
% \resumeItemListEnd

% \vspace{3pt}
% \projectHeading {Refriendo}{TypeScript, React, CSS, Figma}{Oct 2021 - Jul 2022}
% \resumeItemListStart
% \resumeItemWithoutTitle{Implemented a cross-platform Event Management Application for both Android and iOS using React and Ionic Capacitor. Designed and iteratively improved the user interface through high-fidelity Figma prototypes and think-aloud usability testing surveys, resulting in a clean, visually appealing design aesthetic}
% \resumeItemListEnd

%-------- AWARDS------------
\shortSection{Awards \& Leadership}
\skill{Sakura Science Fellow}{Received a fully-funded scholarship from the Japan Science and Technology Agency to participate in a research internship at Ritsukeikan University, Japan}
\skill{Teaching Assistant}{Demonstrated strong leadership and team collaboration skills as a Graduate Teaching Assistant for three courses: Foundations of Software Engineering, Software Requirements \& Interaction Design and Introduction to Graduate Studies}
% \skill{Graduate Teaching Assistant}{Foundations of Software Engineering (CMU - Spring 2023)}
% \skill{Student Leader}{ECE Graduate Organization (CMU - Fall 2022)}
% \skill{Youth Icon (title)}{Won first place in a national public speaking competition among 5000 participants in March 2019}

\end{document}


